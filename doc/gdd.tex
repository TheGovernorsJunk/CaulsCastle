\documentclass{article}
\title{Metroid Clone: The Game}
\author{Tanner Evins}
\date{November 13, 2015}
\usepackage{url}

\begin{document}
\maketitle

\section{Hello!}
If you're reading this, you're one of the fortunate souls chosen to help create the next cutting edge video game (or something close). We're lucky to have a varied group of talented folks. If a bunch of hobbyist writers, artists, and programmers---all of whom game themselves---can't make a game of their own, I'm not sure who can.

Our first project is a relatively simple one: 2D Metroidvania-style platformer. Yes, this game will be wonderfully derivative, but that's okay; it's intentionally so. To my knowledge, none of us have made a game before, so it would behoove us to limit the variables of our project as much as we can. Making a game where we don't have to worry about innovating gameplay mechanics will allow us to become comfortable in our duties as well as working with each other. After this, we can move onto more ambitious projects.

Let's make a game, y'all!

\section{Metroid Review}
The game is a (Super) Metroid clone. We'll have a lot of creative freedom as far as world-building, dialogue, and artwork goes, but the \emph{mechanics} of the game are straight from Metroid.

\subsection*{Elements}

There's no better exposition on the mechanics of (Super) Metroid play than the game itself---which you can do fairly easily with emulation or the Wii U's virtual console---but I'll attempt to enumerate the key elements here:

\begin{itemize}
  \item{\textbf{Exploration.} Metroid games all begin with Samus landing on a foreign world. She's left to her own devices in a strange environment. It's up to you, the player, to discover and explore the world and keep a mental map of it as you go. More on this later.}
  \item{\textbf{Item Discovery.} Along the way, the player encounters items that upgrade Samus's abilities. She doesn't begin with the morph ball ability, for example---she acquires it. Having to discover items instead of starting with them in your arsenal ties into the exploration aspect rather nicely. More often than not, a new item will allow you to access previously unavailable areas. The morph ball enables Samus traverse tiny tunnels.

    Now imagine if the player began the game with the morph ball. The player would be free to travel these tunnels as soon as they're encountered. Delaying the acquisition of the morph ball forces the player to mentally mark those tunnels as significant for a later time. This enforces the player's habit of keeping track of what the world looks like while playing. Very satisfying.}
  \item{\textbf{Nonlinearity.} The original NES Metroid begins famously with the player having to go, not right, but left to progress through the game. This broke the precedent established by platformers like Mario where the player continues right until completing the level. In Metroid, the player goes left, right, up, and down; no direction is privileged as the ``right'' way to go (pun intended).}
\end{itemize}

\subsection*{Two Levels of Play}
I've implied this in the section above, but I'll state it explicitly here: \emph{The essence of Metroid gameplay lies in its two simultaneous levels of play: (1) The immediate situation that demands that you respond to the dangers  and puzzles in the room/area, and (2) the high-level tracking of where you are in the world and where you need to go next.} These two levels work together to sustain the players interest to complete the game.

If you take any given room in Metroid, there's only so many things that can hold your interest. Yes, there will be a few enemies that block your way, but you've likely encountered their type before and know how to deal with them. If you haven't, then you deal with them once and can handle them any time in the future. If a Metroid game were to consist in a series of superficially connected rooms where the player overcame repetitive situation after repetitive situation, then the game would suffer for it.

This is not the way Metroid games work though. The rooms are connected much more deeply. I mentioned above that having the player \emph{discover} items forces the player to make a mental map of how the world appears as a whole.

See an odd looking wall? Maybe it's breakable. I'll remember that. Oh, I've just discovered missiles! I should go back and see if I can destroy it. Praise the sun, I can! Now I'm off to a new area. And since I wasn't able to do that before, I feel that I'm progressing.

It is this kind of gameplay that gives Metroid games their enduring appeal. Combat gameplay has unquestionably improved since the days of the SNES, but players today can still appreciate the high-level thinking required to progress through a Metroid game. It bestows on the world a holistic quality not found in its peers. A lot of good games today take their cue from Metroid in this respect. (Dark Souls and Bloodborne come to mind in particular.)

For a more in-depth overview of this aspect of Metroid, I highly recommend reading ``The Invisible Hand of Super Metroid'' on Gamasutra.\footnote{\url{http://www.gamasutra.com/blogs/HugoBille/20120114/90903/The_Invisible_Hand_of_Super_Metroid.php}}

\section{Creative Freedom in Your Domain}

\end{document}
